% this file is called up by the header file
% content in this file will be fed into the main document
% ----------------------- paths to graphics ------------------------
\graphicspath{{figures/}}

% ----------------------- contents start here ------------------------


\chapter{Conclusion and Outlook}\label{chap:conclusion_outlook}

It is shown that it is possible to evolve effective gaits with machine learning techniques, in this case, genetic algorithms.
This is an important insight because it has several advantages over manually designed gaits.
Manually designed gaits are static in a sense that they cannot be adjusted to a certain environment (e.g. obstacles or different undergrounds).
They always have to be readjusted manually.
On the other side, evolved gaits using genetic algorithms could be implemented in such a way that they learn on-line.
This means that the robot is not stuck to a fixed gait, but can learn steadily after each step.
This has the advantage that it can adjust its gait to a new surface or obstacles.

Since it took a very long time to find an adequate set of parameters and test each set over several hundreds of generations, we most certainly have not found the most efficient evolved gait.
More training time and parameter adjustments should be done in future works.
After this first essential step several topics would be of interest to study in more detail: 
Until now, the gaits are evolved in a robot simulation. 
These gaits could be transferred to the physical robot and be fine-tuned.
The robot could be trained on different surfaces (in simulation and real life) with the goal that the robot automatically adapts its gait to overcome different obstacles.
The ALLBOT comes with different foot shapes.
Training the robot on these different shapes would allow to exhibit foot specific gaits.
Each foot shape could be tested on different surfaces to see if some shapes are more suitable for specific surfaces.
Furthermore, additional legs could be added to the robot to create for example a hexapedal (6-legged) robot.

